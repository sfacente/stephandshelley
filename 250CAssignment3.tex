\documentclass[]{article}
\usepackage{lmodern}
\usepackage{amssymb,amsmath}
\usepackage{ifxetex,ifluatex}
\usepackage{fixltx2e} % provides \textsubscript
\ifnum 0\ifxetex 1\fi\ifluatex 1\fi=0 % if pdftex
  \usepackage[T1]{fontenc}
  \usepackage[utf8]{inputenc}
\else % if luatex or xelatex
  \ifxetex
    \usepackage{mathspec}
  \else
    \usepackage{fontspec}
  \fi
  \defaultfontfeatures{Ligatures=TeX,Scale=MatchLowercase}
\fi
% use upquote if available, for straight quotes in verbatim environments
\IfFileExists{upquote.sty}{\usepackage{upquote}}{}
% use microtype if available
\IfFileExists{microtype.sty}{%
\usepackage{microtype}
\UseMicrotypeSet[protrusion]{basicmath} % disable protrusion for tt fonts
}{}
\usepackage[margin=1in]{geometry}
\usepackage{hyperref}
\hypersetup{unicode=true,
            pdftitle={HW 3},
            pdfauthor={Steph Holm and Shelley Facente},
            pdfborder={0 0 0},
            breaklinks=true}
\urlstyle{same}  % don't use monospace font for urls
\usepackage{longtable,booktabs}
\usepackage{graphicx,grffile}
\makeatletter
\def\maxwidth{\ifdim\Gin@nat@width>\linewidth\linewidth\else\Gin@nat@width\fi}
\def\maxheight{\ifdim\Gin@nat@height>\textheight\textheight\else\Gin@nat@height\fi}
\makeatother
% Scale images if necessary, so that they will not overflow the page
% margins by default, and it is still possible to overwrite the defaults
% using explicit options in \includegraphics[width, height, ...]{}
\setkeys{Gin}{width=\maxwidth,height=\maxheight,keepaspectratio}
\IfFileExists{parskip.sty}{%
\usepackage{parskip}
}{% else
\setlength{\parindent}{0pt}
\setlength{\parskip}{6pt plus 2pt minus 1pt}
}
\setlength{\emergencystretch}{3em}  % prevent overfull lines
\providecommand{\tightlist}{%
  \setlength{\itemsep}{0pt}\setlength{\parskip}{0pt}}
\setcounter{secnumdepth}{0}
% Redefines (sub)paragraphs to behave more like sections
\ifx\paragraph\undefined\else
\let\oldparagraph\paragraph
\renewcommand{\paragraph}[1]{\oldparagraph{#1}\mbox{}}
\fi
\ifx\subparagraph\undefined\else
\let\oldsubparagraph\subparagraph
\renewcommand{\subparagraph}[1]{\oldsubparagraph{#1}\mbox{}}
\fi

%%% Use protect on footnotes to avoid problems with footnotes in titles
\let\rmarkdownfootnote\footnote%
\def\footnote{\protect\rmarkdownfootnote}

%%% Change title format to be more compact
\usepackage{titling}

% Create subtitle command for use in maketitle
\newcommand{\subtitle}[1]{
  \posttitle{
    \begin{center}\large#1\end{center}
    }
}

\setlength{\droptitle}{-2em}

  \title{HW 3}
    \pretitle{\vspace{\droptitle}\centering\huge}
  \posttitle{\par}
    \author{Steph Holm and Shelley Facente}
    \preauthor{\centering\large\emph}
  \postauthor{\par}
      \predate{\centering\large\emph}
  \postdate{\par}
    \date{3/8/2019}


\begin{document}
\maketitle

\section{Parametric Survival
Analysis}\label{parametric-survival-analysis}

\subsection{\texorpdfstring{Question 1: Assuming \emph{Weibull}
distributed event
times,}{Question 1: Assuming Weibull distributed event times,}}\label{question-1-assuming-weibull-distributed-event-times}

\subsubsection{\texorpdfstring{a) Write out the \emph{general}
expression, not substituting any estimated values, and clearly defining
any parameters and distributions for any random terms, the
\textbf{log-hazard function}, including the complete baseline
hazard.}{a) Write out the general expression, not substituting any estimated values, and clearly defining any parameters and distributions for any random terms, the log-hazard function, including the complete baseline hazard.}}\label{a-write-out-the-general-expression-not-substituting-any-estimated-values-and-clearly-defining-any-parameters-and-distributions-for-any-random-terms-the-log-hazard-function-including-the-complete-baseline-hazard.}

\emph{Assuming} \(T \sim\) \emph{Weibull, then the log-hazard function
is expressed as:} \[
log[h_0(t|\boldsymbol{x})] = log(p * t^{p-1}) + \beta_0 +\boldsymbol{x}\beta
\] \emph{where} \(p\) \emph{is the shape parameter,}
\(log(p * t^{p-1}) + \beta_0\) \emph{is the log-baseline hazard, and}
\(\boldsymbol{x}\beta\) \emph{is the log hazard ratio, with Weibull
distribution.}

\vspace{6pt}

\subsubsection{\texorpdfstring{b) Repeat the above for the
\textbf{log-time function} (in the accelerated failure time
metric).}{b) Repeat the above for the log-time function (in the accelerated failure time metric).}}\label{b-repeat-the-above-for-the-log-time-function-in-the-accelerated-failure-time-metric.}

\emph{Again assuming} \(T \sim\) \emph{Weibull, the log-\textbf{time}
function is expressed as:} \[
log(T) = \alpha_0 + \boldsymbol{x}\alpha + \sigma \times \epsilon^\ast
\] \emph{where} \(\alpha_0\) \emph{is the baseline time-to-event,}
\(\boldsymbol{x}\alpha\) \emph{is the log-time ratio,} \(\sigma\)
\emph{is the variance-type constant term (also known as} \(p\),
\emph{unrestricted with Weibull specificiations), and} \(\epsilon^\ast\)
\emph{is the error term, which with a Weibull model approximates a G
distribution (0,1).}

\vspace{6pt}

\subsubsection{c) For an arbitrary covariate, show the expression of the
hazard ratio from the proportional hazards Weibull model as a function
of parameters from the accelerated failure time expression of the same
model.}\label{c-for-an-arbitrary-covariate-show-the-expression-of-the-hazard-ratio-from-the-proportional-hazards-weibull-model-as-a-function-of-parameters-from-the-accelerated-failure-time-expression-of-the-same-model.}

\[h^{AFT}(t|\boldsymbol{x}) = \frac{t^{\frac{1}{\sigma}} - 1}{\sigma}exp(-(\alpha_0 + \boldsymbol{x}\alpha)/\sigma)\]

\emph{Note that this hazard function will take the ratio of \(h^{AFT}\)
but will not necessary be proportional as in the log-hazard function
expressed in part (a)}.

\vspace{12pt}

\subsection{Question 2: Complete the following table using the results
from your
analyses.}\label{question-2-complete-the-following-table-using-the-results-from-your-analyses.}

\begin{longtable}[]{@{}lllllll@{}}
\toprule
\begin{minipage}[b]{0.10\columnwidth}\raggedright\strut
BMI\strut
\end{minipage} & \begin{minipage}[b]{0.19\columnwidth}\raggedright\strut
Cox HR (95\% CI)\strut
\end{minipage} & \begin{minipage}[b]{0.05\columnwidth}\raggedright\strut
Cox \(SE_\beta\)\strut
\end{minipage} & \begin{minipage}[b]{0.19\columnwidth}\raggedright\strut
Exponential HR (95\% CI)\strut
\end{minipage} & \begin{minipage}[b]{0.05\columnwidth}\raggedright\strut
Exp \(SE_\beta\)\strut
\end{minipage} & \begin{minipage}[b]{0.19\columnwidth}\raggedright\strut
Weibull HR (95\% CI)\strut
\end{minipage} & \begin{minipage}[b]{0.05\columnwidth}\raggedright\strut
Weibull \(SE_\beta\)\strut
\end{minipage}\tabularnewline
\midrule
\endhead
\begin{minipage}[t]{0.10\columnwidth}\raggedright\strut
\textless{}18.5\strut
\end{minipage} & \begin{minipage}[t]{0.19\columnwidth}\raggedright\strut
\tt{1.26} (\tt{0.79}, \tt{2.01})\strut
\end{minipage} & \begin{minipage}[t]{0.05\columnwidth}\raggedright\strut
\tt{0.24}\strut
\end{minipage} & \begin{minipage}[t]{0.19\columnwidth}\raggedright\strut
\tt{0.83} (\tt{0.52}, \tt{1.32})\strut
\end{minipage} & \begin{minipage}[t]{0.05\columnwidth}\raggedright\strut
\tt{0.24}\strut
\end{minipage} & \begin{minipage}[t]{0.19\columnwidth}\raggedright\strut
\tt{0.88} (\tt{0.66}, \tt{1.16})\strut
\end{minipage} & \begin{minipage}[t]{0.05\columnwidth}\raggedright\strut
\tt{0.14}\strut
\end{minipage}\tabularnewline
\begin{minipage}[t]{0.10\columnwidth}\raggedright\strut
18.5 - \textless{}25\strut
\end{minipage} & \begin{minipage}[t]{0.19\columnwidth}\raggedright\strut
\emph{ref}\strut
\end{minipage} & \begin{minipage}[t]{0.05\columnwidth}\raggedright\strut
-\strut
\end{minipage} & \begin{minipage}[t]{0.19\columnwidth}\raggedright\strut
\emph{ref}\strut
\end{minipage} & \begin{minipage}[t]{0.05\columnwidth}\raggedright\strut
-\strut
\end{minipage} & \begin{minipage}[t]{0.19\columnwidth}\raggedright\strut
\emph{ref}\strut
\end{minipage} & \begin{minipage}[t]{0.05\columnwidth}\raggedright\strut
-\strut
\end{minipage}\tabularnewline
\begin{minipage}[t]{0.10\columnwidth}\raggedright\strut
25 - \textless{}30\strut
\end{minipage} & \begin{minipage}[t]{0.19\columnwidth}\raggedright\strut
\tt{1.06} (\tt{0.95}, \tt{1.19})\strut
\end{minipage} & \begin{minipage}[t]{0.05\columnwidth}\raggedright\strut
\tt{0.06}\strut
\end{minipage} & \begin{minipage}[t]{0.19\columnwidth}\raggedright\strut
\tt{0.93} (\tt{0.83}, \tt{1.05})\strut
\end{minipage} & \begin{minipage}[t]{0.05\columnwidth}\raggedright\strut
\tt{0.06}\strut
\end{minipage} & \begin{minipage}[t]{0.19\columnwidth}\raggedright\strut
\tt{0.96} (\tt{0.9}, \tt{1.03})\strut
\end{minipage} & \begin{minipage}[t]{0.05\columnwidth}\raggedright\strut
\tt{0.03}\strut
\end{minipage}\tabularnewline
\begin{minipage}[t]{0.10\columnwidth}\raggedright\strut
\(\ge\) 30\strut
\end{minipage} & \begin{minipage}[t]{0.19\columnwidth}\raggedright\strut
\tt{1.54} (\tt{1.32}, \tt{1.78})\strut
\end{minipage} & \begin{minipage}[t]{0.05\columnwidth}\raggedright\strut
\tt{0.24}\strut
\end{minipage} & \begin{minipage}[t]{0.19\columnwidth}\raggedright\strut
\tt{0.69} (\tt{0.59}, \tt{0.8})\strut
\end{minipage} & \begin{minipage}[t]{0.05\columnwidth}\raggedright\strut
\tt{0.08}\strut
\end{minipage} & \begin{minipage}[t]{0.19\columnwidth}\raggedright\strut
\tt{0.78} (\tt{0.71}, \tt{0.85})\strut
\end{minipage} & \begin{minipage}[t]{0.05\columnwidth}\raggedright\strut
\tt{0.05}\strut
\end{minipage}\tabularnewline
\bottomrule
\end{longtable}

\vspace{12pt}

\subsection{Question 3: Answer the following
questions}\label{question-3-answer-the-following-questions}

\subsubsection{a) Which model estimates the relationships of interest
most precisely? Justify your
answer.}\label{a-which-model-estimates-the-relationships-of-interest-most-precisely-justify-your-answer.}

\emph{The Weibull model estimates the relations of interest most
precisely, because as a parametric model the Weibull provides better
statistical efficiency than the Cox model (evidenced by tighter CIs in
the table from Question 2); the Weibull also allows the baseline hazard
to vary over the entire time period (the scale of the shape parameter}
\(p\) \emph{is allowed to vary), making it more flexible than the
exponential model, which restricts} \(p = 1\).

\vspace{2pt}

\subsubsection{b) Based on the likelihood ratio test, what parameter
from the model you outlined in Q1 is being evaluated? Based on the
results of this test, would you select the exponential or Weibull model?
Justify your
answer.}\label{b-based-on-the-likelihood-ratio-test-what-parameter-from-the-model-you-outlined-in-q1-is-being-evaluated-based-on-the-results-of-this-test-would-you-select-the-exponential-or-weibull-model-justify-your-answer.}

\emph{The likelihood ratio test produces the following output:}

\begin{verbatim}
##                                                Terms Resid. Df    -2*LL
## 1 as.factor(bmi_cat) + age + male + factor(cursmoke)      4408 14695.21
## 2 as.factor(bmi_cat) + age + male + factor(cursmoke)      4407 14292.94
##   Test Df Deviance     Pr(>Chi)
## 1      NA       NA           NA
## 2    =  1 402.2752 1.760543e-89
\end{verbatim}

\emph{From this we can see that the p-value for} \(H_0 : p = 1\)
\emph{(exponential) vs.} \(H_A : \sigma \neq 1\) \emph{is essentially
zero; therefore we reject the exponential model and select the Weibull
model.}

\vspace{2pt}

\subsubsection{c) Using the output from the Weibull model, calculate the
time ratio comparing individuals with BMI \textgreater{} 30 to those
with BMI 18.5-\textless{}25 (no need for confidence interval). Interpret
this parameter. Does this agree or not with the corresponding hazard
ratio?}\label{c-using-the-output-from-the-weibull-model-calculate-the-time-ratio-comparing-individuals-with-bmi-30-to-those-with-bmi-18.5-25-no-need-for-confidence-interval.-interpret-this-parameter.-does-this-agree-or-not-with-the-corresponding-hazard-ratio}

\emph{The time ratio (TR) for individuals with BMI \textgreater{} 30
compared to those with BMI 18.5-\textless{}25 is} \(e^{-0.248}\) = .781.
\emph{This means that being obese (BMI \textgreater{} 30) multiplies
survival time by by .781, or decreases survival time by about 22\%.}

\vspace{2pt}

\section{Competing Risks}\label{competing-risks}

\subsection{Question 4: Conceptually, what is the difference between the
KM failure estimate for CVD death and the estimated CIF for CVD death?
What does the comparison of these curves tell you about the risk of the
competing
event?}\label{question-4-conceptually-what-is-the-difference-between-the-km-failure-estimate-for-cvd-death-and-the-estimated-cif-for-cvd-death-what-does-the-comparison-of-these-curves-tell-you-about-the-risk-of-the-competing-event}

\emph{The KM failure estimate for CVD death is the risk of all-cause
mortality at time} \(t\), \emph{for people with CVD (compared to those
without). The estimated CIF for CVD death represents the joint
probability of mortality \textbf{from CVD}, by time} \(t\).
\emph{Comparison of the curves tells you that as days of follow-up
increase, subjects begin to die from causes other than CVD in higher
proportion (i.e.~the curves begin to diverge around 50 days of
follow-up, with the KM failure function continuing to demonstrate
greater probability of mortality than the CIF for the remainder of
follow-up).}

\vspace{6pt}

\subsection{Question 5: In a competing risks analysis, briefly (1-2
sentences each) define in words the following
terms:}\label{question-5-in-a-competing-risks-analysis-briefly-1-2-sentences-each-define-in-words-the-following-terms}

\subsubsection{Cause-specific hazard}\label{cause-specific-hazard}

\emph{A cause-specific hazard is the instantaneous (very short-term)
rate of failure \textbf{for event* \(j\)} }among those who have not yet
experienced the event of interest* \textbf{or} \emph{a competing event
prior to} \(t\).

\vspace{6pt}

\subsubsection{Subdistribution hazard}\label{subdistribution-hazard}

\emph{The subdistribution hazard is the instantaneous (very short-term)
rate of failure for event} \(j\) \emph{among those alive at time} \(t\)
\emph{or who experienced a competing event before time} \(t\).

\vspace{12pt}

\subsection{Question 6: Complete the following table for the
cause-specific hazard ratios (csHRs) and subdistribution HRs (sHRs) you
estimated.}\label{question-6-complete-the-following-table-for-the-cause-specific-hazard-ratios-cshrs-and-subdistribution-hrs-shrs-you-estimated.}

\begin{longtable}[]{@{}lllll@{}}
\toprule
\begin{minipage}[b]{0.10\columnwidth}\raggedright\strut
BMI\strut
\end{minipage} & \begin{minipage}[b]{0.20\columnwidth}\raggedright\strut
CVD csHR (95\% CI)\strut
\end{minipage} & \begin{minipage}[b]{0.20\columnwidth}\raggedright\strut
Other csHR (95\% CI)\strut
\end{minipage} & \begin{minipage}[b]{0.19\columnwidth}\raggedright\strut
CVD sHR (95\% CI)\strut
\end{minipage} & \begin{minipage}[b]{0.19\columnwidth}\raggedright\strut
Other sHR (95\% CI)\strut
\end{minipage}\tabularnewline
\midrule
\endhead
\begin{minipage}[t]{0.10\columnwidth}\raggedright\strut
18.5 - \textless{}25\strut
\end{minipage} & \begin{minipage}[t]{0.20\columnwidth}\raggedright\strut
1\strut
\end{minipage} & \begin{minipage}[t]{0.20\columnwidth}\raggedright\strut
1\strut
\end{minipage} & \begin{minipage}[t]{0.19\columnwidth}\raggedright\strut
1\strut
\end{minipage} & \begin{minipage}[t]{0.19\columnwidth}\raggedright\strut
1\strut
\end{minipage}\tabularnewline
\begin{minipage}[t]{0.10\columnwidth}\raggedright\strut
25 - \textless{}30\strut
\end{minipage} & \begin{minipage}[t]{0.20\columnwidth}\raggedright\strut
\tt{0.89} (\tt{0.31}, \tt{2.53})\strut
\end{minipage} & \begin{minipage}[t]{0.20\columnwidth}\raggedright\strut
\tt{0.54} (\tt{0.19}, \tt{1.51})\strut
\end{minipage} & \begin{minipage}[t]{0.19\columnwidth}\raggedright\strut
\tt{1.17} (.40, 3.5)\strut
\end{minipage} & \begin{minipage}[t]{0.19\columnwidth}\raggedright\strut
\tt{0.41} (.15, 1.1)\strut
\end{minipage}\tabularnewline
\begin{minipage}[t]{0.10\columnwidth}\raggedright\strut
\(\ge\) 30\strut
\end{minipage} & \begin{minipage}[t]{0.20\columnwidth}\raggedright\strut
\tt{2.36} (\tt{0.79}, \tt{7.02})\strut
\end{minipage} & \begin{minipage}[t]{0.20\columnwidth}\raggedright\strut
\tt{0.25} (\tt{0.06}, \tt{0.96})\strut
\end{minipage} & \begin{minipage}[t]{0.19\columnwidth}\raggedright\strut
\tt{3.3} (1.2, 8.9)\strut
\end{minipage} & \begin{minipage}[t]{0.19\columnwidth}\raggedright\strut
\tt{0.18} (.07, .45)\strut
\end{minipage}\tabularnewline
\bottomrule
\end{longtable}

\vspace{12pt}

\subsection{Question 7: From the above table, explain how the pattern
you observe in the csHRs is consistent with the pattern in the sHRs.
(Consider the relationship between the csHR and
sHRs.)}\label{question-7-from-the-above-table-explain-how-the-pattern-you-observe-in-the-cshrs-is-consistent-with-the-pattern-in-the-shrs.-consider-the-relationship-between-the-cshr-and-shrs.}


\end{document}
